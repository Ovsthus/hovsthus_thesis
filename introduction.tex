The end of Moore's law is a challenge that chipmakers and manufacturers have known of for at least a decade. It is a driving factor for the increase in \textit{System on Chip} (SoC) complexity. When the number of transistors per $cm^2$ cannot be increased at the same rate, other alternatives are used to increase performance. Among such alternatives are multiple cores and hardware accelerators. As a result todays SoC's must consist of more and more subsystems of increasing size which subsequently lead to a increasingly vast design space. Over the years much development has been made in describing complex hardware systems at the \textit{Electronic System Level} (ESL). There are some well established methods in using software to represent hardware, a notable one being SystemC with \textit{Transaction Level Modelling} (TLM). The strength of TLM is flexiility and speed, the latter being a result of abstraction. Simulating a large design at the ESL takes orders of magnitude less time than at the \textit{Register Transfer Level} (RTL). Many design decisions can be hedged at this early design stage, along with the detection of numerous bugs. These ESL descriptions already lead to a substantial decrease of the time to market of new products. However, despite these strengths it is the RTL that remains the point of reference for creating the golden design model. The reason for this is the semantic gap between the two abstraction layers ESL and RTL. There is simply not enough trust in the ESL description so functional correctness must be fully verified at the RTL. In spite of much progress in both simulation-based and formal verification techniques, verification at the RTL is the main bottleneck in the SoC design flow. \\
\newline
Some recent research proposes a methology to bridge the semantic gap between the two abstraction layers \cite{2014-UrdahlStoffel.etal}. This research is further developed into a new method of hardware design. With this new method; \textit{Property Driven Development} (PDD), a system can be fully verified at the ESL and a sound relationship between the ESL and RTL is generated in the form of interval properties \cite{pddref}. This research is the foundation that this thesis is based on, and it will be elaborated further in chapter \ref{ch:theory}. Due to reasons that will be explained in chapter \ref{ch:theory} section \ref{sec:ppa} one need to follow a certain set of rules for PDD when describing the system at the ESL. The focus of this thesis is to find a way to represent the complicated bus architecture AMBA-AHB for PDD. Furthermore, to reduce future design efforts a generator is made to enable flexible reuse of this work. The most notable related work is obviously the case study that is the basis of this thesis. The case study is concerned with describing a Wishbone bus architecture at the ESL at two different levels of abstraction. Both ESL descriptions have interval properties generated, and each their RTL designed based on these properties. The case study shows that a higher level of abstraction leads to substantially faster simulation speed. Furthermore, the properties from the abstract design are then refined to hold on the less abstract design showing the flexibility of PDD. This feature is what this thesis is based on, representing complex multi level hardware using an abstract ESL. The case study can be found in \cite{pddref} and its project files in \cite{descam}. The Wishbone bus implementation is a single master non-pipelined design whereas the AMBA-AHB is a multi-master pipelined architecture. The full implementation and its challenges is described in chapter \ref{ch:impl}.          




 


