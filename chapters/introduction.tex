The semiconductor industry has always been striving to increase computer performance. In the past, increasing the transistor density was the way to achieve this performance gain. The rapid increase in transistor density per chip area has lead to higher heat dissipation than what effectively can be cooled. This power wall is a driving factor for the increase in \textit{System on Chip} (SoC) complexity. When the number of transistors per $cm^2$ cannot be increased at the same rate, other alternatives are used to increase performance. Among such alternatives are multiple cores and hardware accelerators. As a result today's SoC's consist of more and more subsystems of increasing size which subsequently lead to a vast increase in design space. In recent years much development has been made in describing complex hardware systems at the \textit{Electronic System Level} (ESL). There are some well established methods in using software to represent hardware, a notable one being SystemC with \textit{Transaction Level Modeling} (TLM). The strength of TLM is flexibility and speed, the latter being a result of abstraction. Simulating a large design at the ESL takes orders of magnitude less time than at the \textit{Register Transfer Level} (RTL). Many design decisions can be hedged at this early design stage, along with the detection of numerous bugs. These ESL descriptions already lead to a substantial decrease of the time to market of new products. However, despite these strengths it is the RTL that remains the point of reference for creating the golden design model. The reason for this is the semantic gap between the two abstraction layers ESL and RTL. There is simply not enough trust in the ESL model so functional correctness must be fully verified at the RTL. In spite of much progress in both simulation-based and formal verification techniques, verification at the RTL is the main bottleneck in the SoC design flow. \par

Some recent research proposes a theoretical framework to bridge the semantic gap between the two abstraction layers \cite{2014-UrdahlStoffel.etal}. This research is further developed into a new methodology of hardware design. With this new methodology; \textit{Property Driven Development} (PDD), a system can be fully verified at the ESL. A sound relationship between the ESL and RTL is ensured by generating interval properties which describe the input/output behavior of the ESL description. The RTL is then designed according to these interval properties \cite{pddref}. This methodology is the foundation that this thesis is based on, and it will be elaborated further in chapter \ref{ch:theory}. Due to reasons that will be explained in chapter \ref{ch:theory} section \ref{sec:ppa} one need to follow a certain set of rules for PDD when describing the system at the ESL. \par 
There are two goals of this thesis ; \\
\begin{enumerate}
 \item Automatically generate an implementation of the bus architecture AMBA-AHB at the ESL and RTL, for a user defined number of masters and slaves. The goal
is to represent the system at the ESL with a high level of abstraction.
 \item Prove that the generated ESL implementation is a sound abstraction of the RTL counterpart.
\end{enumerate}

Generators like these further reduce the required design effort. Design time is focused on the system modules and the hardware designers need only provide them with a simple interface to the generated bus. The ability to generate the bus for any configuration of masters and slaves offer great flexibility in design space exploration. The designer can simulate the design at the ESL and it is proven to be compatible with the AMBA-AHB architecture. Currently, the design has some latency, split transfers are not implemented and burst transfer is only shown as a demonstration. However, this thesis serves as a starting point to representing arbitrated bus architectures using the PDD methodology.   


 


