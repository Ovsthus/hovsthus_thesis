In Sec.\ref{sec:sum} of this chapter, the work presented in chapter \ref{ch:impl} and \ref{ch:design} is summarized. Some thoughts on the implementation and areas of further research are given in Sec.~\ref{sec:outl}. The conclusion of this thesis is given in Sec.~\ref{sec:concl}.

\section{Summary}
\label{sec:sum}
This thesis proposes a generator which generates corresponding models of an AMBA-AHB bus architecture at the Electronic System Level and Register Transfer Level for a selected number of masters and slaves. The ESL description is comprised of two SystemC-PPA modules, one of the master agent and one of the bus matrix, which are used to generate complete sets of operation properties which are automatically refined to prove that the SystemC-PPA modules are sound abstractions of their RTL counterparts. The ESL model and agents implement a subset of the AHB protocol, namely single transfers with split and retry responses disabled. \par
An emulated port handles the parallel nature of the interaction with the master agent interfaces. This port is implemented using existing SystemC-PPA ports. This emulated port ensures correct functional simulation and a set of properties which can be proven on the bus matrix. A generated property set cannot be proven for the emulated port interface itself which leaves a gap in verification. A simple arbitration scheme, namely fixed-priority arbitration is chosen such that soundness with respect to the system level is achieved through careful design of the port. \par
The choice of fixed-priority arbitration leaves lower priority masters vulnerable to starvation. The thesis addresses this problem and provides a graphical comparison of starvation between the two models. \par
Simulation models were generated to compare the two models for a range of configurations. The experimental results show the simulation time is reduced by orders of magnitude in the ESL model compared to the RTL model. \par 
Burst transfers are only shown as a proof of concept due to time constraints, since the level of complexity for implementing this is high. The proof of concept is a four-beat incremental burst implemented on a system comprised of a single master and slave. This demonstration shows that burst transfers can be implemented with some limitations. Proving completeness on a design with burst transfers is feasible. \par
 

\section{Outlook}
\label{sec:outl}
Much has been achieved in this work with regards to generating an AMBA-AHB at the ESL and RTL. The individual modules which represent the master agent and bus matrix at the ESL are formally proven to be sound abstractions of their RTL counterparts. Due to the required emulated port interface it still remains to formally prove that the ESL model is a sound abstraction of the RTL model with respect to the system as a whole. Furthermore, fixed-priority arbitration results in the challenge of starvation, which require further research to handle.    

\subsection{Suggested improvements}
\label{sec:impr}
\subsubsection{Signal name map}
The plugin which refines the properties rely on matching signal names between the models and conditional statements to refine the signal macros. A suggested improvement is to rely on a map which returns the appropriate refinement based on the abstract signal name. The generator would rely on the same map to name the signals such that the model can be configured with more flexibility


\subsection{Further research}
\subsubsection{Formally proven $agent\leftrightarrow bus$ interface}
More research is required on how to formally prove the interface between the master agent and bus matrix. This may be a complex but general solution which implements different arbitration schemes within a SystemC-PPA port interface that can be used in any arbitrated bus architecture. It can be envisioned as a port reader which can have many writers connected. The reader issues two notify events, one is unconditional and signals bus is ready, the other is conditional and signals grant. How arbitration is tackled remains to be determined. It will likely still need to be represented in both the port and in the module implementing the port reader. Perhaps a tool can verify that the arbitration scheme in the model is equivalent to the arbitration scheme in the port. This way, more complex arbitration schemes like round robin can be safely implemented.  

\subsubsection{Monitor starvation}
It was shown that starvation is not necessarily over approximated in the simulation model. A monitor could be implemented, but it requires more research into how this is done. Theoretically, it can monitor the amount of simulation time, or number of context switches within each master between each transfer.
If it is not guaranteed that the master which is granted in the ESL model will be granted in the RTL model, the simulation issues a warning or halts the simulation. Some restrictions must apply, such that any context switch or simulation increment which is counted by the monitor equates to at least one clock cycle in the RTL. Furthermore, the level of abstraction of the operation within the slave need a measurement or constraint. The monitor must know if an operation
takes one or more cycles at the RTL in relation to the ESL.   


\section{Conclusion}
\label{sec:concl}
Two goals were set in chapter \ref{ch:intro} of this thesis. This work shows the requirements for generating an abstract representation of the AMBA-AHB with SystemC-PPA. It is possible to generate the system at the ESL and RTL and automatically refine the properties such that they hold on the design. The final system implements single read and write transactions with split and retry response types disabled. Burst transactions are shown to be possible through a proof of concept. Simulations of both models show that all configurations of the ESL model are at least ten times faster than any configuration of the RTL model. More work needs to be carried out to implement the burst transactions in the generator. More research remains to formally prove soundness of the interface between the master agent and bus matrix. More research is required into the specifics of the monitor to tackle the challenge of starvation. 






     



